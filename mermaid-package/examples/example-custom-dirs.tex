\documentclass[a4paper,12pt]{article}

% Пример использования кастомных директорий для mermaid.sty
% Компиляция: xelatex --shell-escape -output-directory=output example-custom-dirs.tex

\usepackage{fontspec}
\usepackage{polyglossia}
\setdefaultlanguage{russian}
\setmainfont{Arial}

\usepackage{graphicx}
\usepackage[hidelinks]{hyperref}

% Используем кастомные директории
% outputdir должен совпадать с -output-directory
% mermaiddir - любое имя для поддиректории
\usepackage[outputdir=output, mermaiddir=diagrams]{mermaid}

\title{Пример использования mermaid.sty v1.0.0}
\author{Демонстрация конфигурируемых директорий}
\date{\today}

\begin{document}

\maketitle

\section{Введение}

Этот документ демонстрирует использование пакета \texttt{mermaid.sty} с кастомными настройками директорий.

\textbf{Конфигурация:}
\begin{itemize}
  \item \texttt{outputdir=output} (вместо дефолтного \texttt{build})
  \item \texttt{mermaiddir=diagrams} (вместо дефолтного \texttt{mermaid-images})
\end{itemize}

\section{Примеры диаграмм}

\subsection{Блок-схема}

\begin{figure}[h]
  \centering
  \begin{mermaidenv}[width=0.6\linewidth]
  graph TD
    A[Начало] --> B{Проверка}
    B -->|Успех| C[Обработка]
    B -->|Ошибка| D[Логирование]
    C --> E[Конец]
    D --> E
  \end{mermaidenv}
  \caption{Пример блок-схемы}
  \label{fig:flowchart}
\end{figure}

\subsection{Диаграмма последовательности}

\begin{figure}[h]
  \centering
  \begin{mermaidenv}[width=0.7\linewidth]
  sequenceDiagram
    participant Client
    participant Server
    participant Database

    Client->>Server: Request
    Server->>Database: Query
    Database-->>Server: Result
    Server-->>Client: Response
  \end{mermaidenv}
  \caption{Взаимодействие компонентов}
  \label{fig:sequence}
\end{figure}

\section{Структура файлов}

После компиляции структура директорий будет следующей:

\begin{verbatim}
project/
├── example-custom-dirs.tex
├── diagrams → output/diagrams/  (симлинк)
└── output/
    ├── example-custom-dirs.pdf
    ├── example-custom-dirs.aux
    ├── example-custom-dirs.log
    └── diagrams/
        ├── mermaid-1.mmd
        ├── mermaid-1.mmd.png
        ├── mermaid-2.mmd
        └── mermaid-2.mmd.png
\end{verbatim}

\section{Заключение}

Гибкая конфигурация позволяет:
\begin{itemize}
  \item Организовать структуру проекта по своему усмотрению
  \item Использовать разные имена директорий для разных документов
  \item Легко интегрироваться в существующие системы сборки
\end{itemize}

\end{document}
