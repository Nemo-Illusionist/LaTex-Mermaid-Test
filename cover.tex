% ==== Параметры (с дефолтами через \providecommand) ====
\providecommand{\MinistryName}{МИНИСТЕРСТВО ОБРАЗОВАНИЯ И НАУКИ\\
ПРИДНЕСТРОВСКОЙ МОЛДАВСКОЙ РЕСПУБЛИКИ}

\providecommand{\UniversityBlockA}{Государственное образовательное учреждение\\
высшего образования}

\providecommand{\UniversityName}{ПРИДНЕСТРОВСКИЙ ГОСУДАРСТВЕННЫЙ УНИВЕРСИТЕТ\\
им.\,Т.\,Г.\,ШЕВЧЕНКО}

\providecommand{\Subject}{«Название предмета»}
\providecommand{\Direction}{<<Код и название направления>>}

\providecommand{\AuthorName}{Фамилия И.О.}
\providecommand{\AuthorPosition}{должность, кафедра}

\providecommand{\City}{Тирасполь}
\providecommand{\Year}{\the\year}

% Опционально: логотип. Пример: \def\LogoPath{logo.png}
% В main.tex можно определить \LogoPath — тогда логотип вставится.
% Ширину можно менять тут:
\providecommand{\LogoWidth}{2.2cm}

% ====== Сам титульный лист (не содержит \documentclass) ======
\begingroup
\thispagestyle{empty}

% Верхний блок (по центру)
\begin{center}

% Логотип (опционально)
\ifdefined\LogoPath
  \includegraphics[width=\LogoWidth]{\LogoPath}\\[0.5cm]
\fi

\small{\MinistryName}\\[0.3cm]
\small{\UniversityBlockA}\\[0.3cm]
\textbf{\small\UniversityName}\\[1.5cm]

\textbf{\Large МЕТОДИЧЕСКИЕ УКАЗАНИЯ}\\[0.2cm]
\small{по дисциплине}\\[0.2cm]
\textbf{\large \Subject}\\[0.7cm]
\small{для студентов направления подготовки}\\[0.2cm]
\textbf{\small \Direction}\\[2cm]
\end{center}

% Блок «Составитель» справа
\vspace*{\fill}
\hfill
\begin{minipage}{0.55\textwidth}
  \raggedleft
  \textbf{Составитель:}\\
  \AuthorName\\
  \AuthorPosition
\end{minipage}
\vspace*{\fill}

% Низ страницы (по центру)
\begin{center}
\City\\
\Year{} г.
\end{center}

\endgroup
\clearpage