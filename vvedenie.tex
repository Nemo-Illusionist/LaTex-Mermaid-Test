\section*{Введение}
\addcontentsline{toc}{section}{Введение}

Дисциплина \Subject\\
является одной из ключевых в подготовке специалистов в области информатики и вычислительной техники. 
Современные программные системы строятся с применением принципов объектно-ориентированного подхода, 
что делает владение данным стилем программирования необходимым элементом профессиональной компетенции будущего инженера-программиста.

Цель изучения дисциплины — формирование у студентов представления об основных принципах 
объектно-ориентированного программирования, а также развитие навыков практической реализации 
объектно-ориентированных решений на языке C\#.

Задачи дисциплины включают:
\begin{itemize}
  \item освоение базовых понятий объектно-ориентированной парадигмы (класс, объект, наследование, полиморфизм);
  \item изучение синтаксиса и особенностей языка C\#;
  \item приобретение навыков разработки и отладки программ на платформе .NET;
  \item формирование опыта проектирования программных систем с использованием принципов инкапсуляции и модульности.
\end{itemize}

Методические указания предназначены для студентов направления подготовки \Direction\\
и могут быть использованы при выполнении лабораторных и практических работ, а также при самостоятельной подготовке к зачёту или экзамену.

Рекомендуется последовательно изучать материал каждой темы, выполнять приведённые упражнения и самостоятельно решать дополнительные задачи.
